% By Ccucumber12
\providecommand{\tightlist}{\setlength{\itemsep}{0pt}\setlength{\parskip}{0pt}}
\setcounter{secnumdepth}{0}

Time Limit: 1 s \\
Memory Limit: 2048 KB
\vspace{-15pt}

\subsection{Problem Description}\label{problem-description}

\textbf{Problem 5 described the bonus (i.e. hard) version of the game. Compared with the basic version in Problem 4, this version is harder by having a ``scoring system'' in the game.} We allocate \textbf{only 20} bonus (extra) points for this problem. That is, your score for HW1 can be as high as $520$. Somehow calling the problem ``bonus'' means we do not expect everyone to try to solve the problem---you are recommended to solve other problems first before visiting this problem.

%This is a hard version of the problem. This version has to calculate the total score of each person.}
\vspace{6pt}

% Little Cucumber is developing "Digital Savage Arena (DSA)", a KoH-like (King of the Hill) video game, with a plan to release the alpha version on April 4th for gathering feedback from players. Unfortunately, he has been struggling with the scoring system for weeks, given his lack of practice in data structure courses. As you have survived the notorious HW0, can you help him with the final piece of the puzzle?

% In DSA, players fight each other in an arena to obtain prestigious honor. The game consists of $N$ rounds and $N$ players. The $i$-th player has an attack power of $a_i$, and they will enter the arena at the beginning of round $i$. When a player enters the arena, they immediately kill all other payers who have a lower attack power. 

% Each round, the player with the highest attack power will gain the title "King of DSA". If two or more players have the same attack power, the one entered earlier will take the honor. However, due to the limited memory, the arena cannot be too big. If the number of alive players in the arena exceeds $M$, a revolution will occur. In this case, the King of DSA will be overthrown and killed by the other $M$ players. 

% \vspace{10pt}
% \textit{(The above is identical to the easy version.)}
% \vspace{10pt}

Our story is exactly the same as Problem 4, except that Little Cucumber decides to add a scoring system to the game. 
%The scoring system is based on the King of the Hill concept. 
After confirming the players that are alive in each round, the player with the highest attack power (i.e. King of DSA) earns $M$ points. The player with the second-highest attack power earns $M-1$ points, and so on. Only the surviving players in the arena are given points. Ties are broken by the order that the players entered the arena---i.e. the player that enters earlier earns more points.

In short, round $i$ of the game executes the following three steps orderly:
\vspace{-6pt}
\begin{enumerate}[leftmargin=50pt]
\tightlist
    \item Player $i$ enters the arena.
    \item Player $i$ kills all other players that hold a lower attack power.
    \item Check if a revolution occurs. If so, the King of DSA is killed.
    \item \textbf{Assign points to the surviving players.}
\end{enumerate}

% For simplicity, each round of the game can be viewed as having 4 phases:
% \vspace{-6pt}
% \begin{enumerate}[leftmargin=50pt]
% \tightlist
%     \item Player $i$ enters the arena.
%     \item Player $i$ kills every other alive player who has a lower attack power. 
%     \item Check if a revolution occurs. 
%     \item Rank and award the alive players. 
% \end{enumerate}

\noindent Given the attack power of the $N$ players, can you help Little Cucumber calculate not only the players that are killed in each round, but also their scores (accumulated points)?

%Given the attack power of the $N$ players, can you help Little Cucumber calculate the death events and their total scores in each round?

\subsection{Input}\label{input}

The first line contains two integers $N$ and $M$---the number of rounds (players) and the arena's maximum capacity. $M$ also means the points given to King of DSA in each round.
The second line contains $N$ integers $a_1, a_2, \dots, a_N$,  which is the attack power of each player, separated by space.

\subsection{Output}\label{output}

The output should consist of $N+1$ lines. The first $N$ lines should begin with the string ``\texttt{Round $i$:}'', where $i$ is the round number. Each of the $N$ lines should include the player(s) that are killed in that round. The last line should begin with the string ``\texttt{Final:}'', followed by the players who are still alive at the end of the game. 

% The output should consist of $N+1$ lines. 
% The first $N$ lines should begin with the string ``\texttt{Round i:}'', where $i$ is the round number. 
% Each of the $N$ lines should include the final scores for the players who died during that round. 
% The last line should begin with the string ``\texttt{Final:}'', followed by the final scores of the players who are still alive at the end of the game.

When outputting the players (killed ones in the first lines and the surviving ones in the last line), print out the players, separated by space, in the \textbf{descending} order of their \textbf{indices} (the round that each of them entered the arena). Each player should be outputted as a string that consists of zir index and total score, separated by a comma, like ``\texttt{index,score}''. 

% and separated by space.
% The information for the killed players and last surviving players on each of the $N+1$ lines should be separated by a space and sorted in \textbf{descending} order based on the players' \textbf{indices}.
% Each piece of the information should consist of the player's index and their total score, separated by a comma, as follows: ``\texttt{index,score}''. 
% Please refer to the sample outputs for a clearer understanding of the expected output format. 

\subsection{Constraints} \label{constraint}
\begin{itemize}
\tightlist
    \item $1 \le N \le 10^6$
    \item $1 \le M \le 10^3$
    \item $1 \le a_i \le 10^9$
\end{itemize}

\subsubsection{Subtask 1 (1 bonus pt)}\label{subtask-1}

\begin{itemize}
\tightlist
\item $1 \le N \le 10^3$
\end{itemize}

\subsubsection{Subtask 2 (9 bonus pts)}\label{subtask-2}

\begin{itemize}
\tightlist
\item It is guaranteed that revolution will never occur. 
\end{itemize}

\subsubsection{Subtask 3 (10 bonus pts)}\label{subtask-3}

\begin{itemize}
\tightlist
\item No other constraints.
\end{itemize}

\newpage

\subsection{Sample Testcases}

\subsubsection{Sample Input 1}\label{sample-input-1}
\begin{verbatim}
10 4
8 5 2 4 7 6 5 3 4 6
\end{verbatim}

\subsubsection{Sample Output 1}\label{sample-output-1}
\begin{verbatim}
Round 1:
Round 2:
Round 3:
Round 4: 3,2
Round 5: 4,2 2,9
Round 6:
Round 7:
Round 8: 1,28
Round 9: 8,1
Round 10: 9,1 7,5
Final: 10,2 6,13 5,21
\end{verbatim}

\subsubsection{Sample Input 2}\label{sample-input-2}
\begin{verbatim}
8 3
5 5 5 5 6 4 4 5
\end{verbatim}

\subsubsection{Sample Output 2}\label{sample-output-2}
\begin{verbatim}
Round 1:
Round 2:
Round 3:
Round 4: 1,9
Round 5: 4,1 3,3 2,7
Round 6:
Round 7:
Round 8: 7,1 6,4
Final: 8,2 5,12
\end{verbatim}